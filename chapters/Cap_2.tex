\chapter{Nombre del Capítulo II (Usualmente Marco Teórico)}\label{chap:background}

Cada capítulo deberá contener una breve introducción que describe en forma rápida el contenido del
mismo. En este capítulo va el marco teórico. (pueden ser dos capítulos de marco teórico)

\section{Sección 1 del Capítulo II}

Un capítulo puede contener $n$ secciones. 

Con respecto a las referencias bibliográficas, se hace de la siguiente manera \cite{Mateos00}, se debe de redactar el texto de forma tal que si se retiraran las referencias, la redacción no debe perjudicarse. Por ejemplo 

\subsubsection{Localización y Mapeo Simultaneos SLAM puramente topológico}
Se puede hacer un sistema SLAM con sólo reconocimiento de lugares, generando un SLAM topológico con representación basada en grafos \cite{Choset_2001}. Se almacena un registro de lugares visitados y cómo se llega de uno a otro (conexiones) sin tener información geométrica explícita.

Este tipo de SLAM es más adecuado para planificación y navegación. La figura \ref{Fig:SLAMTopological} ilustra este tipo de SLAM 
    
\begin{figure}[ht]
\centering
% placeholder figure (original content omitted in template)
\fbox{\parbox[c][4cm][c]{8cm}{\centering SLAM puramente topológico (figura de ejemplo)}}
\caption{SLAM puramente topológico basado en grafos}
\label{Fig:SLAMTopological}
\end{figure}


\subsection{Sub Sección}

Una sección puede contener $n$ sub secciones.\cite{Galante01}

\subsubsection{Sub sub sección}

Una sub sección puede contener n sub sub secciones.

\section{Recomendaciones generales de escritura}
Un trabajo de esta naturaleza debe tener en consideración varios aspectos generales:

\begin{itemize}
\item Ir de lo genérico a lo específico. Siempre hay que considerar que el lector podría ser alguien no muy familiar con el tema
	y la lectura debe serle atractiva.
\item No redactar frases muy largas. Si las frases tienen más de 2 líneas continuas es probable que la lectura sea dificultosa.
\end{itemize}

Cada capítulo excepto el primero debe contener, al finalizar, una sección de consideraciones que enlacen el presente capítulo con el siguiente.
