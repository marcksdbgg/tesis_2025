\chapter{Propuesta}\label{chap:proposal}

En este capítulo se desarrolla toda la propuesta realizada a través de la investigación. Sigue la misma estructura del capítulo anterior.

El título del capítulo es flexible de acuerdo a cada tesis. Algunos títulos sugeridos podrían ser:

\begin{itemize}
\item Algoritmo X: nuestra propuesta.
\item Modelo MRLO
\end{itemize}

Este título debe de ser definido junto a su asesor de tesis. Consúltelo en su sala de clase.
\chapter{Propuesta}
\label{chap:proposal}

En este capítulo se presenta un análisis exhaustivo del estado del arte en \textit{Retrieval‑Augmented Generation} (RAG) y se propone una arquitectura adaptada a las necesidades del proyecto.  El objetivo es ofrecer un panorama actualizado de las tendencias más recientes (2023–2025) y sentar las bases para diseñar un sistema de consulta visual interactiva que combine recuperación eficiente y generación de lenguaje natural.  Se describen los motivos que justifican el uso de RAG, su evolución, las variantes existentes (offline y como servicio), los marcos de desarrollo de código abierto y los modelos base que se emplean actualmente.  Finalmente, se proponen lineamientos para una implementación local que elimine las dependencias de servicios en la nube y se adapte a un entorno de un único usuario.

\section{Motivación y limitaciones de los modelos generativos}

Los modelos de lenguaje de gran escala (\textit{LLM}) han demostrado capacidades notables en tareas de generación y comprensión de texto.  Sin embargo, estos modelos presentan limitaciones inherentes: su conocimiento está ``congelado'' en la fecha de corte del corpus de entrenamiento y no incorporan datos privados o recientes.  Además, generan salidas probabilísticas que pueden contener errores u ``alucinaciones'' \cite{pinecone2025guide}.  Un análisis de Pinecone señala que los modelos de fundación carecen de profundidad en dominios especializados, no incluyen datos privados y no proporcionan citas de sus fuentes, lo que erosiona la confianza de los usuarios【916670030940373†L32-L92】.  Estas limitaciones justifican la búsqueda de mecanismos que conecten a los LLM con fuentes de información externas y actualizadas.

\section{Definición y componentes de RAG}

\textit{Retrieval‑Augmented Generation} es una técnica que combina la búsqueda de información con la generación de lenguaje natural.  Según el tutorial de SingleStore \cite{singlestore2025tutorial}, RAG es un modelo que integra los procesos de recuperación y generación para mejorar la precisión de los LLM, mitigando las alucinaciones al incorporar datos externos relevantes.  La guía de Pinecone resume este enfoque en cuatro componentes: (1) ingestión de datos de autoridad en una base vectorial; (2) recuperación de información relevante mediante búsquedas densa y dispersa; (3) aumento del mensaje combinando la consulta del usuario con la información recuperada; y (4) generación de la respuesta por el LLM utilizando el contexto aportado【916670030940373†L118-L133】.  Este flujo permite proporcionar respuestas actualizadas, trazables y contextualizadas.

\section{Evolución y taxonomía de RAG}

Una revisión reciente \cite{sharma2025survey} analiza la evolución de RAG y propone una taxonomía basada en la localización de las innovaciones.  Se distinguen arquitecturas centradas en el recuperador, centradas en el generador y sistemas híbridos.  Los sistemas \emph{retriever‑centric} mejoran la búsqueda mediante técnicas de redensificación, filtrado y búsqueda híbrida; los \emph{generator‑centric} ajustan la decodificación y la integración del contexto; mientras que los híbridos coordinan ambas etapas para equilibrar precisión y flexibilidad【589007710703999†L74-L93】.  El estudio también revisa avances como la re‑clasificación dinámica de resultados, el uso de filtros de diversidad y los mecanismos de seguridad ante entradas adversarias【589007710703999†L960-L981】.

El preprint también enumera los modelos base empleados en los experimentos de referencia, entre ellos \textit{LLaMA 2}, \textit{LLaMA 3}, \textit{GPT‑3.5/4}, \textit{Vicuna}, \textit{Mistral}, \textit{Mixtral}, \textit{Gemini} y \textit{Gemma}【589007710703999†L1016-L1020】.  Esta diversidad muestra que la técnica es agnóstica al modelo subyacente y permite combinar LLM comerciales o de código abierto con recuperadores densos y dispersos.

\subsection{Mejoras y optimizaciones}

Las investigaciones recientes proponen varias mejoras para incrementar la eficacia del pipeline RAG.  Entre las más destacadas se encuentran:

\begin{itemize}
  \item \textbf{Búsqueda híbrida y filtrado de contexto}: combinar vectores densos con índices dispersos (BM25) para capturar tanto la similitud semántica como coincidencias exactas de términos【916670030940373†L238-L250】.  Asimismo, se aplican filtros de diversidad o estrategias de \textit{maximal marginal relevance} para reducir la redundancia y mejorar la cobertura.
  \item \textbf{Reordenamiento mediante re‑rankers}: después de recuperar los primeros $k$ documentos, se emplean modelos \textit{cross‑encoder} (por ejemplo, BGE Reranker o modelos similares) que asignan un puntaje de similitud más preciso a cada par consulta‑documento.  Pinecone señala que el re‑ordenamiento se efectúa tras combinar los resultados de búsqueda densa y dispersa para devolver los fragmentos más relevantes【916670030940373†L246-L251】.
  \item \textbf{Optimización de la generación}: se exploran estrategias como la generación condicional con mezcla de expertos, la reincorporación iterativa de evidencias o la restricción de la decodificación para reducir alucinaciones.  El estudio de Sharma \cite{sharma2025survey} discute técnicas de control de alucinación, adaptabilidad a dominios y seguridad frente a ataques adversarios【589007710703999†L960-L981】.
\end{itemize}

\section{RAG en entornos offline}

La mayoría de los sistemas RAG se ofrece como servicios en la nube, pero existe un creciente interés por implementaciones locales que preserven la privacidad y funcionen sin conexión a internet.  El blog de Alex Ho describe una experiencia de \emph{RAG offline} utilizando Ollama, un entorno que permite ejecutar modelos de lenguaje localmente mediante GPU【246311036369843†L87-L96】.  El autor narra cómo empleó \textit{Llama 2} y \textit{Llama 3} en su portátil y desarrolló un pequeño CLI que responde preguntas sobre notas Markdown almacenadas en su computadora【246311036369843†L104-L110】.  Para preparar la base de datos personalizada, el flujo consiste en: leer los archivos, dividir el texto en fragmentos de longitud fija (con solapamiento para no perder contexto), generar embeddings con un modelo local (por ejemplo \texttt{nomic‑embed‑text}) y almacenarlos junto con metadatos en una base vectorial como Chroma【246311036369843†L204-L225】.  Una vez indexados los vectores, el usuario puede generar consultas, las cuales se embeben y comparan con el índice; los fragmentos recuperados se envían a un modelo local para elaborar la respuesta.  Este enfoque evita el coste de las API comerciales y ofrece control total sobre los datos.

\section{RAG como servicio (RAGaaS)}

Otra tendencia es el \textit{RAG‑as‑a‑Service}.  Las empresas de infraestructura ofrecen plataformas que gestionan la ingesta, el indexado, la recuperación y la generación a través de una API.  Un artículo de Vlad Koval expone que la promesa de RAGaaS consiste en permitir a los clientes ``conectarse'' a un servicio que maneja el escalado, las actualizaciones y la observabilidad de la recuperación, de modo que la organización pueda concentrarse en el caso de uso【430126204644756†L53-L62】.  Este enfoque acelera la experimentación y resulta atractivo para equipos con recursos limitados【430126204644756†L82-L99】.  No obstante, el autor advierte que se sacrifica el control sobre aspectos clave como la segmentación de documentos, la actualización de embeddings o la afinación de los re‑rankers【430126204644756†L68-L76】.  RAGaaS puede funcionar bien en pruebas piloto o para chatbots de soporte, pero en dominios regulados o cuando los costes de escala son elevados conviene optar por una implementación propia【430126204644756†L106-L114】.  La página de Microsoft sobre Azure AI Search muestra un ejemplo de RAGaaS orientado a empresas: el patrón consiste en enviar la consulta del usuario a un servicio de búsqueda que devuelve resultados relevantes, los cuales se inyectan en el prompt del modelo generativo; se destacan la necesidad de estrategias de indexado, ajuste de relevancia, integración con modelos de embeddings y consideraciones de seguridad【316064638127775†L56-L78】.

\section{Marcos de desarrollo y herramientas de código abierto}

El ecosistema de RAG cuenta con numerosas bibliotecas y plataformas que facilitan la construcción de aplicaciones.  A continuación se resumen los principales marcos de código abierto según el informe de Designveloper \cite{designveloper2025frameworks} y la guía de LangCopilot \cite{langcopilot2025guide}:

\subsection{Haystack}

Haystack, desarrollada por Deepset, es un marco modular para construir aplicaciones de preguntas y respuestas y flujos RAG.  Permite conectar diversas tecnologías como OpenAI, Chroma y Hugging Face gracias a su arquitectura flexible【149667013130509†L319-L343】.  Incluye componentes pre‑construidos para generadores, recuperadores y embedders, así como un sistema de \textit{pipelines} serializables que se pueden desplegar en Kubernetes.  También ofrece una interfaz web (Hayhooks) para exponer pipelines como servicios y herramientas de trazado y evaluación【149667013130509†L344-L352】.

\subsection{LangChain}

LangChain es una plataforma composable que facilita el desarrollo de agentes y cadenas de procesamiento para LLMs.  Ofrece un IDE visual, plantillas de prompts y múltiples herramientas para resolver las complejidades técnicas【149667013130509†L359-L376】.  Sus componentes incluyen modelos de chat, recuperadores, cargadores de documentos, almacenes vectoriales y utilidades de ingeniería de prompts【149667013130509†L366-L371】.  Las cadenas son secuencias de componentes reutilizables que pueden combinarse y monitorizarse, lo que habilita flujos complejos y memoria conversacional.  Además, se integra con otras herramientas de la familia Lang (LangSmith, LangGraph) que proporcionan depuración, orquestación y despliegue a escala【149667013130509†L383-L387】.

\subsection{LlamaIndex}

LlamaIndex (antes \textit{GPT Index}) se centra en la ingesta, indexado y consulta de datos empresariales.  Simplifica la creación de asistentes de conocimiento mediante componentes modulares de embedding, carga, indexado y consulta【149667013130509†L393-L401】.  Aunque su foco principal es el texto, incorpora soporte emergente para RAG multimodal a través de LlamaExtract y servicios de la plataforma LlamaCloud que permiten extraer datos estructurados de documentos complejos【149667013130509†L401-L415】.  También ofrece funciones avanzadas para memoria conversacional, flujos de múltiples pasos y herramientas de evaluación y observabilidad【149667013130509†L407-L410】.

\subsection{RAGFlow}

RAGFlow es un motor de RAG orientado a la comprensión profunda de documentos y a la generación de respuestas con citas.  Soporta la integración de modelos mediante APIs y permite desplegar modelos locales mediante Ollama o Xinference, brindando flexibilidad entre rendimiento, coste y privacidad【149667013130509†L422-L440】.  Dispone de un editor visual de nodos para diseñar flujos, incorpora funcionalidades de construcción de grafos de conocimientos, extracción automática de palabras clave y preguntas, y posibilita la observabilidad detallada de cada paso mediante integración con Langfuse【149667013130509†L430-L439】.

\subsection{txtAI}

txtAI es un framework que integra búsqueda semántica, orquestación de LLM y flujos de lenguaje.  Proporciona una base de embeddings que combina índices densos y dispersos y soporta grafos y estructuras relacionales【149667013130509†L446-L453】.  Sus principales ventajas son: compatibilidad con embeddings multimodales (texto, audio, imágenes, video), capacidad para combinar pipelines en flujos complejos y disponibilidad de APIs web y MCP en varios lenguajes.  Permite el despliegue local y escalable utilizando contenedores【149667013130509†L446-L465】.

\subsection{Cognita}

Cognita es un marco de Truefoundry orientado a crear sistemas RAG escalables.  Incluye parsers para diferentes tipos de datos, cargadores de datos desde diversas fuentes, embedders que soportan modelos de OpenAI y Cohere, re‑rankers de última generación y un administrador de consultas capaz de gestionar múltiples peticiones y escalar los recursos automáticamente【149667013130509†L472-L489】.  Su arquitectura modular facilita la integración de diferentes componentes y garantiza el cumplimiento de normas de seguridad y privacidad.

\subsection{Dify}

Dify es un marco de bajo código que ofrece una interfaz de arrastrar y soltar para desarrollar flujos agentes.  Su \textit{Backend‑as‑a‑Service} gestiona la complejidad del desarrollo y transforma los flujos construidos en servicios accesibles mediante MCP【149667013130509†L496-L501】.  Permite integrar modelos de OpenAI, Hugging Face, Grok o DeepSeek y ampliar las capacidades mediante \textit{plugins} (Slack, QRCode, xAI, AWS, etc.)【149667013130509†L503-L507】.  Además, soporta capacidades multimodales y estrategias inteligentes para orquestar tareas complejas.

\subsection{Elección de framework}

Elegir un marco RAG depende de múltiples factores.  Designveloper sugiere considerar la licencia (preferir Apache 2.0 o MIT), el nivel de modularidad, la facilidad de uso (orientado a código o interfaz visual), la capacidad para manejar documentos largos y las integraciones disponibles【149667013130509†L518-L538】.  Por ejemplo, Haystack y LangChain destacan en modularidad y soporte para backends diversos, mientras que RAGFlow y Dify ofrecen interfaces visuales orientadas a usuarios no desarrolladores.  También es necesario evaluar la ventana de contexto soportada y la capacidad de procesar documentos extensos【149667013130509†L595-L599】.

\section{Modelos de embeddings, bases vectoriales y LLMs}

Un sistema RAG requiere seleccionar modelos de embeddings, almacenes de vectores y modelos generativos apropiados.  Existen modelos ligeros como \textit{all‑MiniLM‑L6‑v2} de Sentence Transformers o \textit{BGE} que generan vectores de dimensión reducida y permiten la búsqueda eficiente.  Para mejorar la precisión se suele añadir un re‑ranker \textit{cross‑encoder}, como los modelos BGE \textit{reranker}, que calculan la similitud entre la consulta y el documento de forma conjunta【916670030940373†L246-L251】.  Las bases vectoriales más utilizadas incluyen FAISS, Chroma, Weaviate y Pinecone.  Chroma es útil para implementaciones locales porque almacena los metadatos en SQLite y los índices en su propia estructura【246311036369843†L218-L225】.  Para la generación, el estado del arte incluye modelos como LLaMA 2/3, Mistral, Mixtral, Vicuna, Gemini, Gemma y GPT‑4, que se pueden ejecutar localmente (mediante Ollama o similares) o a través de APIs comerciales【589007710703999†L1016-L1020】.

\section{Propuesta de implementación local}

Con base en la revisión anterior y las necesidades del proyecto, se propone construir un sistema RAG local que elimine la multi‑tenencia y las dependencias de servicios en la nube.  La arquitectura estaría compuesta por los siguientes módulos:

\begin{enumerate}
  \item \textbf{Ingestión y procesamiento}: se cargan documentos desde el sistema de archivos, se extrae texto (\textit{PDF}, \textit{DOCX}, \textit{HTML}, \textit{TXT}), se dividen en fragmentos de tamaño configurable y se generan embeddings con un modelo local (p.\,ej., \textit{all‑MiniLM} o \textit{nomic‑embed‑text}).  Los vectores y metadatos se almacenan en una base vectorial local (FAISS o Chroma) y en una base relacional ligera como SQLite.
  \item \textbf{Búsqueda y re‑rankeo}: para cada consulta se genera un embedding, se recuperan los fragmentos más relevantes mediante búsqueda densa e híbrida, se combinan los resultados y se reordenan con un modelo \textit{cross‑encoder}.  También se pueden aplicar estrategias de diversidad para evitar redundancia.
  \item \textbf{Generación}: los fragmentos seleccionados se incorporan en el prompt de un modelo de lenguaje local (\textit{Llama 3} o \textit{Mistral}) que produce la respuesta en español.  Se incluyen citas que apuntan a las fuentes consultadas.
  \item \textbf{Interfaz de usuario}: se propone una interfaz web local o una aplicación de línea de comandos que permita subir documentos, iniciar consultas y visualizar las respuestas junto con sus fuentes.  La autenticación se simplifica a un único usuario local y no se expone la API fuera del equipo.
\end{enumerate}

Esta solución aprovecha los beneficios de RAG (precisión, trazabilidad y flexibilidad) mientras preserva la privacidad de los datos y reduce los costes operativos.  La modularidad de los componentes permite sustituir el modelo de embeddings, la base vectorial o el LLM según las necesidades futuras.

\section{Conclusiones}

La técnica de \textit{Retrieval‑Augmented Generation} ha evolucionado rápidamente en los últimos años y se ha convertido en un pilar para construir sistemas conversacionales precisos y verificables.  La revisión realizada muestra que los modelos generativos por sí solos sufren de alucinaciones y obsolescencia, mientras que RAG los complementa con recuperación de información actual y contextualizada【916670030940373†L145-L156】.  Existen numerosas variantes y optimizaciones que permiten adaptar la técnica a diferentes dominios, desde implementaciones locales con \textit{software} libre hasta plataformas de \textit{RAG‑as‑a‑Service}.  Los marcos de código abierto como Haystack, LangChain, LlamaIndex, RAGFlow, txtAI, Cognita y Dify facilitan el desarrollo y despliegue de estas soluciones, cada uno con ventajas y limitaciones específicas【149667013130509†L518-L538】.  Finalmente, se propone una arquitectura local que integra estos conceptos y responde a las necesidades de este proyecto, proporcionando una base sólida para la consulta visual interactiva de grandes volúmenes de datos multidimensionales.